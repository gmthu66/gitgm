\documentclass[11pt]{article}
\usepackage[margin=1in]{geometry}                % See geometry.pdf to learn the layout options. There are lots.
\geometry{letterpaper}                   % ... or a4paper or a5paper or ... 
%\geometry{landscape}                % Activate for for rotated page geometry
%\usepackage[parfill]{parskip}    % Activate to begin paragraphs with an empty line rather than an indent


\renewcommand{\aa}{\mathbb{A}}
\newcommand{\cc}{\mathbb{C}}
\newcommand{\rr}{\mathbb{R}}
\newcommand{\pp}{\mathbb{P}}
\newcommand{\hh}{\mathbb{H}}
\newcommand{\qq}{\mathbb{Q}}
\newcommand{\zz}{\mathbb{Z}}
\newcommand{\ff}{\mathbb{F}}
\newcommand{\kk}{\mathbb{K}}
\renewcommand{\gg}{\mathbb{G}}
\newcommand{\nn}{\mathbb{N}}
\renewcommand{\tt}{\mathbb{T}}

\newcommand{\C}{\mathcal{C}}
\newcommand{\U}{\mathcal{U}}
\newcommand{\I}{\mathcal{I}}
\renewcommand{\H}{\mathcal{H}}
\renewcommand{\O}{\mathcal{O}}
\newcommand{\E}{\mathcal{E}}
\newcommand{\F}{\mathcal{F}}
\newcommand{\G}{\mathcal{G}}
\renewcommand{\P}{\mathcal{P}}
\renewcommand{\S}{\mathcal{S}}
\newcommand{\Q}{\mathcal{Q}}
\newcommand{\T}{\mathcal{T}}
\renewcommand{\L}{\mathcal{L}}
\newcommand{\M}{\mathcal{M}}
\newcommand{\X}{\mathcal{X}}

\newcommand{\sH}{\mathscr{H}}
\newcommand{\sD}{\mathscr{D}}
\newcommand{\sE}{\mathscr{E}}
\newcommand{\sL}{\mathscr{L}}
\newcommand{\sQ}{\mathscr{Q}}
\newcommand{\sX}{\mathscr{X}}




\usepackage{graphicx}
\usepackage{amssymb, amsmath}
\usepackage{epstopdf}
\DeclareGraphicsRule{.tif}{png}{.png}{`convert #1 `dirname #1`/`basename #1 .tif`.png}

\title{18.06 Recitation 13}
\author{Isabel Vogt}
\date{\today}                                           % Activate to display a given date or no date

\begin{document}
\maketitle
%\section{}
%\subsection{}

\begin{enumerate}


\item Let 
\[A = \begin{pmatrix} 2 & -1 & & &  \\-1 & 2 & -1 && \\ & -1 & 2 & -1 & \\ && -1 & 2 & -1 \\ && &-1 & 2 \end{pmatrix}. \]
be a tridiagonal matrix.  You can imagine that $A$ is the $5 \times 5$ version of a family of tridiagonal matrices of size $n \times n$.

\begin{enumerate}

\item  How can you use Gaussian elimination to solve $Ax = b$ in linear time?

\item What is special about the $LU$ factorization of $A$?  What is the pattern of nonzero entries.

\item What is the pattern of nonzero entries of $A^{-1}$?


\end{enumerate}


\item Suppose that $A$ is a $4 \times 4$ matrix with singular value decomposition $AV = U \Sigma$ given by 
\[A \begin{pmatrix}  & && 1/\sqrt{2} \\ v_1 & v_2& v_3& 0 \\  &&& 0 \\  & && -1/\sqrt{2} \end{pmatrix} = \begin{pmatrix} 1/3 & && 0 \\ 2/3 & u_2& u_3& 0 \\ 0 &&& 1 \\ 2/3 & && 0 \end{pmatrix} \begin{pmatrix} 1 &&& \\ & 0.2 && \\ && 0.00001 & \\ &&& 0 \end{pmatrix}.\]

\begin{enumerate}
\item What is the SVD of $A^T$?

\item What is the rank of $A$?  Give an expression for a smaller rank matrix that approximates $A$.

\item Is $Ax = \begin{pmatrix} 1 \\ -2 \\ 1 \\ 1 \end{pmatrix}$ solvable?

\item  What is the matrix $P$ that projects $\rr^4$ onto $N(A)$?

\item  Let $B = AA^T$ and let $y = \begin{pmatrix} 1 \\ 0 \\ 0 \\ 0 \end{pmatrix}$.  As $n \to \infty$, what do you know about $B^n y$?

\item Suppose that you wanted to use the power method to compute $v_1$.  How would you set this up?  How does the error behave after $k$ iterates?

\end{enumerate}


\item Let 
\[A = \begin{pmatrix} 2 & -1 & & & -1 \\-1 & 2 & -1 && \\ & -1 & 2 & -1 & \\ && -1 & 2 & -1 \\ -1&& &-1 & 2 \end{pmatrix}. \]
\begin{enumerate}

\item  What are the eigenvectors of $A$?  What special properties do they have?

\item What are eigenvalues corresponding to these eigenvectors?


\end{enumerate}


\item  Assume that $A$ is diagonalizable with eigenvalues
\[|\lambda_1| > |\lambda_2 | > \cdots > |\lambda_n|. \]
We may find $\lambda_1$ and a corresponding eigenvector $v_1$ using the power method.  Let $P$ be the projection matrix onto the orthogonal complement to the span of $v_1$.  

Now suppose that we want to find $\lambda_2$ and a corresponding eigenvector $v_2$.  Consider the following algorithm
\begin{enumerate}
\item[1.] Start with a random vector $x_0$.
\item[2.] Iteratively set
\[ x_k =  \frac{PAx_{k-1}}{||PAx_{k-1}||}. \] 
\end{enumerate}

\begin{enumerate}

\item  Assuming that $A$ is normal (the eigenvectors for distinct eigenvalues are orthogonal) what does this converge to?

\item  If $A$ is not normal what can you say about the vector $y = \lim_{k \to \infty} x_k$ that this converges to?
Let 
\[Q = \begin{pmatrix} & \\ v_1 & y \\ & \end{pmatrix} \]
What do you know about the matrix $Q$?

\item Write $AQ = QS$ for some matrix $S$.  Do you know anything special about $S$ (i.e. does it have a particular pattern of zero entries)?

\item What is $\lambda_2$ and a choice of $v_2$?  (hint: use $S$!) 


\end{enumerate}

\item Suppose that $A$ is a $3 \times 3$ matrix that is not diagonalizable, but has a basis of generalized eigenvectors $\{x_1, j_1, x_2 \}$ where
\[Ax_1  = \lambda_1 x_1, \qquad Aj_1 = \lambda_1 j_1 + x_1, \qquad Ax_2 = \lambda_2 x_2.\]

\begin{enumerate}

\item Is the matrix $A + I$ diagonalizable?  Give the eigenvalues and a basis of (generalized) eigenvectors.

\item  What are the eigenvalues of the matrix $A^n$?  What is a basis of (generalized) eigenvectors of $A^n$ in terms of $x_1, j_1, x_2$?

\item What are the eigenvalues of the matrix $e^{At}$?  What is a basis of (generalized) eigenvectors of $e^{At}$ in terms of $x_1, j_1, x_2$?

\end{enumerate}









\end{enumerate}
\end{document}  