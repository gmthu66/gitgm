\documentclass[11pt]{article}
\usepackage[margin=0.5in]{geometry}                % See geometry.pdf to learn the layout options. There are lots.
\geometry{letterpaper}                   % ... or a4paper or a5paper or ... 
%\geometry{landscape}                % Activate for for rotated page geometry
%\usepackage[parfill]{parskip}    % Activate to begin paragraphs with an empty line rather than an indent


\renewcommand{\aa}{\mathbb{A}}
\newcommand{\cc}{\mathbb{C}}
\newcommand{\rr}{\mathbb{R}}
\newcommand{\pp}{\mathbb{P}}
\newcommand{\hh}{\mathbb{H}}
\newcommand{\qq}{\mathbb{Q}}
\newcommand{\zz}{\mathbb{Z}}
\newcommand{\ff}{\mathbb{F}}
\newcommand{\kk}{\mathbb{K}}
\renewcommand{\gg}{\mathbb{G}}
\newcommand{\nn}{\mathbb{N}}
\renewcommand{\tt}{\mathbb{T}}

\newcommand{\C}{\mathcal{C}}
\newcommand{\U}{\mathcal{U}}
\newcommand{\I}{\mathcal{I}}
\renewcommand{\H}{\mathcal{H}}
\renewcommand{\O}{\mathcal{O}}
\newcommand{\E}{\mathcal{E}}
\newcommand{\F}{\mathcal{F}}
\newcommand{\G}{\mathcal{G}}
\renewcommand{\P}{\mathcal{P}}
\renewcommand{\S}{\mathcal{S}}
\newcommand{\Q}{\mathcal{Q}}
\newcommand{\T}{\mathcal{T}}
\renewcommand{\L}{\mathcal{L}}
\newcommand{\M}{\mathcal{M}}
\newcommand{\X}{\mathcal{X}}

\newcommand{\sH}{\mathscr{H}}
\newcommand{\sD}{\mathscr{D}}
\newcommand{\sE}{\mathscr{E}}
\newcommand{\sL}{\mathscr{L}}
\newcommand{\sQ}{\mathscr{Q}}
\newcommand{\sX}{\mathscr{X}}


\usepackage{makecell}
\usepackage{pdflscape}
\usepackage{graphicx}
\usepackage{amssymb, amsmath}
\usepackage{epstopdf}
\DeclareGraphicsRule{.tif}{png}{.png}{`convert #1 `dirname #1`/`basename #1 .tif`.png}

\title{18.06 Recitation 13}
\author{Isabel Vogt}
\date{\today}                                           % Activate to display a given date or no date

\begin{document}
%\maketitle
%\section{}
%\subsection{}

\begin{center}
\textbf{The Four Fundamental Subspaces} \\
$A$ is $m \times n$ matrix of rank $r$ \\
$R = \operatorname{rref}(A)$,
$M = \operatorname{rref}(A^T)$, \qquad
The SVD of $A$ is $U \Sigma V^T$
\end{center}

\begin{tabular}{ l | l }
Nullspace: $N(A)$ & Row space: $C(A^T)$  \\[5pt] \hline \\
$\bullet$ definition: & $\bullet$ definition: \\ \\ \\ \\
$\bullet$ subspace of \underline{\phantom{aaaaaaaaa}} of dimension \underline{\phantom{aaaaa}} \phantom{aaaaaaa} & $\bullet$ subspace of \underline{\phantom{aaaaaaaaa}} of dimension \underline{\phantom{aaaaa}} \phantom{aaaaaaa} \\ \\ \\
$\bullet$ orthogonal complement: & $\bullet$ orthogonal complement: \\ \\ \\
$\bullet$ relation to $N(R)$ or $N(M^T)$ : & $\bullet$ relation to $C(R^T)$ or $C(M)$ : \\ \\ \\
\makecell[l]{$\bullet$ if $N(A) = 0$ then we say that $A$ has\\ \\ \qquad  \underline{\phantom{aaaaaaaaaaaaaaaaaaaaaaaaaaaaaaaaa}}} & \makecell[l]{$\bullet$ if $C(A^T) = \rr^n$ then we say that $A$ has\\ \\ \qquad  \underline{\phantom{aaaaaaaaaaaaaaaaaaaaaaaaaaaaaaaaa}}} \\ \\
$\bullet$ an orthonormal basis given by: & $\bullet$ an orthonormal basis given by: \\  \\ \\ \hline \hline \\
Column space: $C(A)$ & Left nullspace: $C=N(A^T)$  \\[5pt] \hline \\
$\bullet$ definition: & $\bullet$ definition: \\ \\ \\ \\
$\bullet$ subspace of \underline{\phantom{aaaaaaaaa}} of dimension \underline{\phantom{aaaaa}} \phantom{aaaaaaa} & $\bullet$ subspace of \underline{\phantom{aaaaaaaaa}} of dimension \underline{\phantom{aaaaa}} \phantom{aaaaaaa} \\ \\ \\
$\bullet$ orthogonal complement: & $\bullet$ orthogonal complement: \\ \\ \\
$\bullet$ relation to $C(R)$ or $C(M^T)$ : & $\bullet$ relation to $N(R^T)$ or $N(M)$ : \\ \\ \\
\makecell[l]{$\bullet$ if $C(A) = \rr^m$ then we say that $A$ has\\ \\ \qquad  \underline{\phantom{aaaaaaaaaaaaaaaaaaaaaaaaaaaaaaaaa}}} & \makecell[l]{$\bullet$ if $N(A^T) = 0$ then we say that $A$ has\\ \\ \qquad  \underline{\phantom{aaaaaaaaaaaaaaaaaaaaaaaaaaaaaaaaa}}} \\ \\
$\bullet$ an orthonormal basis given by: & $\bullet$ an orthonormal basis given by: \\ \\ \\  \\
\end{tabular}

\pagebreak 
\begin{center}
\textbf{Number of Operations}
\end{center}

\begin{enumerate}
\item Multiplying an $m \times n$ matrix $A$ by an $n$ component vector $v$: \\ \\ \\
\item Multiplying an $m \times n$ matrix $A$ by an $n \times r$ matrix $B$: \\ \\ \\

\item Computing the dot product of two $m$ component vectors $v$ and $w$: \\ \\ \\

\item Finding the inverse of a matrix $A$ using the Gauss-Jordan method: \\ \\ \\
 
\item Finding the $LU$ factorization of an $m \times m$ matrix $A$ using Gaussian elimination: \\ \\ \\

\item Finding the $LU$ factorization of an $m \times m$ upper-triangular matrix $A$ using Gaussian elimination: \\ \\ \\

\item Finding the $LU$ factorization of an $m \times m$ matrix $A$ such that $a_{ij} =0$ if $i > j+1$ using Gaussian elimination: \\ \\ \\

\item Finding the $LU$ factorization of an $m \times m$ matrix $A$ such that $a_{ij} =0$ if $i > j+1$ or $j > i +1$ using Gaussian elimination: \\ \\ \\

\item Solving an upper-triangular system $Ux = b$, where $U$ is an $m \times m$ upper-triangular matrix and $b$ is an $m$ component vector:\\ \\ \\

\item Solving an upper-triangular system $Ux = b$, where $U$ is an $m \times m$ upper-bidiagonal matrix (so $u_{ij} =0$ if $i > j$ or $j > i +1$) and $b$ is an $m$ component vector:

\end{enumerate}

\pagebreak

\begin{center}
\textbf{Linear Algebraic Algorithms}
\end{center}

\noindent \textsc{Gaussian Elimination:}
\begin{itemize}
\item Purpose: \\ \\
\item Input: \\ 
\item Any assumptions on the input? \\
\item Output: \\ \\
\item Approximate \# of operations (based on size of input): \\ \\
\item Factorizations? \\
\item Pattern of nonzero entries if input is:
\begin{itemize}
\item tridiagonal
\item lower triangular
\item upper triangular
\end{itemize}
\end{itemize}

\noindent \textsc{Gram-Schmidt:}
\begin{itemize}
\item Purpose: \\ \\
\item Input: \\ 
\item Any assumptions on the input? \\
\item Output: \\ \\
\item Approximate \# of operations (based on size of input): \\ \\
\item Factorizations? \\
\item Pattern of nonzero entries if input is:
\begin{itemize}
\item tridiagonal
\item lower triangular
\item upper triangular
\end{itemize}
\end{itemize}


\begin{landscape}

\begin{tabular}{ c | c | c | c | c | c | c }
The matrix is... & Definition & Special about \textbf{eigenvalues} & Special about \textbf{eigenvectors} & Special about \textbf{det} & Factorizations & Consequences \\ \hline \hline
\makecell{\phantom{aaaaaaaa} \\ invertible \\ \phantom{aaaaaaaa} \\ \phantom{aaaaaaaa}}  & & & & & & \\ \hline
\makecell{\phantom{aaaaaaaa} \\ \phantom{aaaaaaaa} \\ projection \\ \phantom{aaaaaaaa} \\ \phantom{aaaaaaaa}}  & & & & & & \\ \hline
\makecell{\phantom{aaaaaaaa}  \\ \phantom{aaaaaaaa} \\ Markov \\ \phantom{aaaaaaaa} \\ \phantom{aaaaaaaa}}  & & & & & & \\ \hline
\makecell{\phantom{aaaaaaaa}  \\ positive \\ Markov \\ \phantom{aaaaaaaa} \\ \phantom{aaaaaaaa}}  & & & & & & \\ \hline
\makecell{ \phantom{aaaaaaaa} \\(real  \\ orthogonal) \\ unitary \\ \phantom{aaaaaaaa}}  & & & & & & \\ \hline
\makecell{\phantom{aaaaaaaa} \\ (real  \\ symmetric) \\ Hermitian \\ \phantom{aaaaaaaa}}  & & & & & & \\ \hline
\makecell{\phantom{aaaaaaaa} \\ (real  anti- \\ symmetric) \\ anti-Hermitian \\ \phantom{aaaaaaaa}}  & & & & & & \\ \hline
\makecell{\phantom{aaaaaaaa} \\ positive/negatie \\  (semi)definite \\ Hermitian \\ \phantom{aaaaaaaa}}  & & & & & & \\ \hline
\end{tabular}



Suppose that $A$ is a matrix with eigenvectors $v_1, \cdots, v_n$ and eigenvalues $\lambda_1, \cdots, \lambda_n$.  Write as much as you know about the eigenvectors and eigenvalues of the following matrices: \\

\begin{tabular}{ c | c | c | c | c | c | c | c  }
& \phantom{aaaa} $-A$  \phantom{aaa} & \phantom{aaaa} $A^{2017}$ \phantom{aaa} & \phantom{aaaa} $A^4 + 3A + I$ \phantom{aaa} & \phantom{aaaa} $A^T$ \phantom{aaa} & \phantom{aaaa} $e^{A^2}$\phantom{aaa} & \phantom{aaaa} $(A + I)^{-1}$ \phantom{aaa} & \phantom{aaaa} $SAS^{-1}$ \phantom{aaa} \\ \hline \hline
\makecell{\phantom{aaaaaaaa} \\\phantom{aaaaaaaa} \\ eigenvalues \\ \phantom{aaaaaaaa} \\ \phantom{aaaaaaaa}} & &&&& & \\ \hline
\makecell{\phantom{aaaaaaaa} \\\phantom{aaaaaaaa} \\ eigenvectors \\ \phantom{aaaaaaaa} \\ \phantom{aaaaaaaa}} &&&&& &
\end{tabular}

\vspace{50pt}

Suppose instead that $A$ is $3 \times 3$ and not diagonalizable, with a basis of generalized eigenvectors $v_1, j_1, v_2$, so that $Aj_1 = \lambda_1 j_1 + v_1$.  Now write down as much as you know about the eigenvalues, (generalized) eigenvectors, and diagonalizability of the following matrices: \\

\begin{tabular}{ c | c | c | c | c | c | c | c  }
& \phantom{aaaa} $-A$  \phantom{aaaa} & \phantom{aaaa} $A^{2}$ \phantom{aaaa} & \phantom{aaaa} $A + I$ \phantom{aaaa} & \phantom{aaaa} $A^T$ \phantom{aaaa} & \phantom{aaaa} $e^{A}$\phantom{aaaa} & \phantom{aaaa} $A^{-1}$ \phantom{aaaa} & \phantom{aaaa} $SAS^{-1}$ \phantom{aaaa} \\ \hline \hline
\makecell{\phantom{aaaaaaaa} \\\phantom{aaaaaaaa} \\ eigenvalues \\ \phantom{aaaaaaaa} \\ \phantom{aaaaaaaa}} & &&& &&  \\ \hline
\makecell{\phantom{aaaaaaaa} \\ (generalized) \\ eigenvectors \\ \phantom{aaaaaaaa} \\ \phantom{aaaaaaaa}} &&&& && \\ \hline
\makecell{\phantom{aaaaaaaa} \\\phantom{aaaaaaaa} \\ diagonalizable? \\ \phantom{aaaaaaaa} \\ \phantom{aaaaaaaa}} &&&& &&
\end{tabular}




\end{landscape}

 \pagebreak
 


Suppose that $A$ is an $(m+1) \times (m+1)$ nondiagonalizable matrix with eigenvalues $\lambda_1$ (with multiplicity 2), $\lambda_2, \dots, \lambda_m$, and a corresponding basis of (generalized) eigenvectors $x_1, j_1, x_2, \dots, x_m$ (so $Aj_1 = \lambda_1 j_1 + x_1$ and $Ax_i = \lambda_i x_i$).



\begin{enumerate}

\item What is the solution to the differential equation $\frac{dx}{dt} = A x$ with initial condition $x(0)$? \\ \\

\item For \emph{generic $x(0)$}, under what conditions does:
\begin{enumerate}
\item $|x(t)| \to \infty$, as $t \to \infty$? \\ \\ \\
\item $|x(t)| \to 0$, as $t \to \infty$? \\ \\ \\
\item $|x(t)| \to$ a nonzero constant, as $t \to \infty$? \\ \\ \\
\item $x(t) \to$ a nonzero constant, as $t \to \infty$? \\ \\ \\
\item $x(t)$ oscillates, as $t \to \infty$? \\ \\ \\
\end{enumerate}

\item For \emph{some choice of $x(0)$}, under what conditions does:
\begin{enumerate}
\item $|x(t)| \to \infty$, as $t \to \infty$? \\ \\ \\
\item $|x(t)| \to 0$, as $t \to \infty$? \\ \\ \\
\item $|x(t)| \to$ a nonzero constant, as $t \to \infty$? \\ \\ \\
\item $x(t) \to$ a nonzero constant, as $t \to \infty$? \\ \\ \\
\item $x(t)$ oscillate, as $t \to \infty$? \\ \\ \\
\end{enumerate}

\item What is the solution to the recurrence $x_{k} = A x_{k-1}$ with initial condition $x_0$? \\ \\

\item For \emph{generic $x_0$}, under what conditions does:
\begin{enumerate}
\item $|x_n| \to \infty$, as $n \to \infty$? \\ \\ \\
\item $|x_n| \to 0$, as $n \to \infty$? \\ \\ \\
\item $|x_n| \to$ a nonzero constant, as $n \to \infty$? \\ \\ \\
\item $x_n \to$ a nonzero constant, as $n \to \infty$? \\ \\ \\
\item $x_n$ oscillates, as $n \to \infty$? \\ \\ \\
\end{enumerate}

\item For \emph{some choice of $x_0$}, under what conditions does:
\begin{enumerate}
\item $|x_n| \to \infty$, as $n \to \infty$? \\ \\ \\
\item $|x_n| \to 0$, as $n \to \infty$? \\ \\ \\
\item $|x_n| \to$ a nonzero constant, as $n \to \infty$? \\ \\ \\
\item $x_n \to$ a nonzero constant, as $n \to \infty$? \\ \\ \\
\item $x_n$ oscillates, as $n \to \infty$? \\ \\ \\
\end{enumerate}

\end{enumerate}











\end{document}  