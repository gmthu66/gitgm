\documentclass[11pt]{article}
\usepackage[margin=1in]{geometry}                % See geometry.pdf to learn the layout options. There are lots.
\geometry{letterpaper}                   % ... or a4paper or a5paper or ... 
%\geometry{landscape}                % Activate for for rotated page geometry
%\usepackage[parfill]{parskip}    % Activate to begin paragraphs with an empty line rather than an indent


\renewcommand{\aa}{\mathbb{A}}
\newcommand{\cc}{\mathbb{C}}
\newcommand{\rr}{\mathbb{R}}
\newcommand{\pp}{\mathbb{P}}
\newcommand{\hh}{\mathbb{H}}
\newcommand{\qq}{\mathbb{Q}}
\newcommand{\zz}{\mathbb{Z}}
\newcommand{\ff}{\mathbb{F}}
\newcommand{\kk}{\mathbb{K}}
\renewcommand{\gg}{\mathbb{G}}
\newcommand{\nn}{\mathbb{N}}
\renewcommand{\tt}{\mathbb{T}}

\newcommand{\C}{\mathcal{C}}
\newcommand{\U}{\mathcal{U}}
\newcommand{\I}{\mathcal{I}}
\renewcommand{\H}{\mathcal{H}}
\renewcommand{\O}{\mathcal{O}}
\newcommand{\E}{\mathcal{E}}
\newcommand{\F}{\mathcal{F}}
\newcommand{\G}{\mathcal{G}}
\renewcommand{\P}{\mathcal{P}}
\renewcommand{\S}{\mathcal{S}}
\newcommand{\Q}{\mathcal{Q}}
\newcommand{\T}{\mathcal{T}}
\renewcommand{\L}{\mathcal{L}}
\newcommand{\M}{\mathcal{M}}
\newcommand{\X}{\mathcal{X}}

\newcommand{\sH}{\mathscr{H}}
\newcommand{\sD}{\mathscr{D}}
\newcommand{\sE}{\mathscr{E}}
\newcommand{\sL}{\mathscr{L}}
\newcommand{\sQ}{\mathscr{Q}}
\newcommand{\sX}{\mathscr{X}}




\usepackage{graphicx}
\usepackage{amssymb, amsmath}
\usepackage{epstopdf}
\DeclareGraphicsRule{.tif}{png}{.png}{`convert #1 `dirname #1`/`basename #1 .tif`.png}

\title{18.06 Recitation 8}
\author{Isabel Vogt}
\date{\today}                                           % Activate to display a given date or no date

\begin{document}
\maketitle
%\section{}
%\subsection{}

\begin{enumerate}

\item The Fibonacci sequence is defined recursively by specifying initial values $F_0=1,  F_1 = 1$, and the relation
\[ F_{n+1} = F_n + F_{n-1}. \]
\begin{enumerate}
\item Given the input vector $v_n = \begin{pmatrix} F_n \\ F_{n-1} \end{pmatrix}$, what is the matrix $A$, so that $Av_n = v_{n+1}$?

Give an expression for $v_{n+1}$ in terms of $A$ and $v_1 = \begin{pmatrix} 1 \\ 0 \end{pmatrix}$.

\item What are the eigenvectors and eigenvalues of $A$?

\item Using the eigenbasis, give an exact formula for $F_n$.  What do you know about $|F_n|$ as $n \to \infty$?

\end{enumerate}

\item Let $A$ be an $m \times m$ matrix and let $A'$ be the matrix obtained from $A$ be reversing the rows and columns:
\[ A = \begin{pmatrix} a_{11} & a_{12} & \dots & a_{1m} \\ a_{21} & a_{22} & \dots & a_{2m} \\  & & \ddots & \\ a_{m1} & a_{m2} & \dots & a_{mm} \end{pmatrix}, \qquad A' = \begin{pmatrix} a_{mm} & a_{m(m-1)} & \dots & a_{m1} \\ a_{(m-1)m} & a_{(m-1)(m-1)} & \dots & a_{(m-1)1} \\  & & \ddots & \\ a_{1m} & a_{1(m-1)} & \dots & a_{11} \end{pmatrix}. \]

\begin{enumerate}
\item When $m=2$, what do you notices about the eigenvalues of $A'$?
\item What is true in general and why?
\end{enumerate}

\item (Strang, Section 10.3, Problem 6)  
\begin{enumerate}
\item For a Markov matrix, show that the sum of the components of $x$ equals the sum of the components of $Ax$.
\item If $Ax =  \lambda x$ with $\lambda \neq 1$, prove that the components of this non-steady eigenvector $x$ add to zero.
\end{enumerate}


\item 

\begin{enumerate}
\item Show that every square matrix is similar to its transpose.  That is, 
\[A^T=SAS^{-1}, \] 
for some invertible matrix $S$.
\item Assuming $A$ is diagonalizable, give a formula for $S$ in terms of the matrix $X$ of eigenvectors of $A$ and the matrix $Y$ of eigenvectors of $A^T$ (which is also diagonalizable).
\end{enumerate}


\end{enumerate}
\end{document}  