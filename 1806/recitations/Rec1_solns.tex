\documentclass[11pt]{article}
\usepackage[margin=1in]{geometry}                % See geometry.pdf to learn the layout options. There are lots.
\geometry{letterpaper}                   % ... or a4paper or a5paper or ... 
%\geometry{landscape}                % Activate for for rotated page geometry
%\usepackage[parfill]{parskip}    % Activate to begin paragraphs with an empty line rather than an indent
\usepackage{graphicx}
\usepackage{amssymb, amsmath}
\usepackage{epstopdf}
\DeclareGraphicsRule{.tif}{png}{.png}{`convert #1 `dirname #1`/`basename #1 .tif`.png}

\title{18.06 Recitation 1}
\author{Isabel Vogt}
\date{September 12, 2017}                                           % Activate to display a given date or no date

\begin{document}
\maketitle
%\section{}
%\subsection{}



\begin{enumerate}

%\item Let $m$ and $n$ be natural numbers (you can think of $m=3$ and $n=2$ if you like).  Let $A$ be an $m \times n$ matrix, $B$ be a $n \times n$ matrix, $v$ be a $1 \times n$ vector, $w$ be a $m \times 1$ vector, and $u$ be a $1 \times m$ vector.  Which of the following make sense (and if so, what is the result)
%
%\begin{enumerate}
%
%\item $A B$
%
%\item $BA$
%
%\item $A v$
%
%\item $A w$
%
%\item $ v A$
%
%\item $v B$
%
%\item $wu$
%
%\item $w v$
%
%\item $u w$
%
%\end{enumerate}
%
%\item If you have two coupled systems of equations
%\begin{align*}
%Bx + Cy &= c \\
%Dx + Ey &= d.
%\end{align*}
%where $B, C, D$, and $E$ are $3 \times 3$ matrices and $x,y,c$ and $d$ are $3$-component vectors, can you write this as a single system of equations,
%\[Az = b, \]
%where $z = \begin{bmatrix} x \\ y \end{bmatrix}$ is the $6$-component vector of $x$ on top of $y$.  What are $A$ and $b$?
%
%\item Describe geometrically why the system of linear equations
%\begin{align*}
%3x + y &= 2\\
%6x + 2y &= 4
%\end{align*}
%has infinitely many solutions.  Describe geometrically why the system of linear equations
%\begin{align*}
%3x + y &= 2\\
%6x + 2y &= 10
%\end{align*}
%has no solutions.
%
%\item ``Do" Gaussian elimination on the matrix
%\[ \begin{bmatrix} 2 & 0 & 0 \\ 4 & 1 & 0 \\ -1 & 6 & 2 \end{bmatrix}.\]
%
\item[5.] What is the $LU$ factorization of the matrix
\[A = \begin{bmatrix} 0 & 1 \\ 2 & 3 \end{bmatrix}? \]

\textbf{Solution:} This matrix doesn't have an $LU$ factorization!  Here's why: suppose that there were a lower triangular matrix
\[L = \begin{bmatrix} x & 0 \\ y & z \end{bmatrix} \]
and an upper triangular matrix 
\[U = \begin{bmatrix} a & b \\ 0 & c \end{bmatrix}\]
so that
\[A = \begin{bmatrix} 0 & 1 \\ 2 & 3 \end{bmatrix} = \begin{bmatrix} x & 0 \\ y & z \end{bmatrix}\begin{bmatrix} a & b \\ 0 & c \end{bmatrix} = LU. \]

In particular, this means that the upper left entry of the matrix $A$ would satisfy
\[0 = x \cdot a + 0 \cdot 0 = xa. \] 
But this means that one of $x$ or $a$ is $0$, as the product of two nonzero numbers is always nonzero.

So either $x$ or $a$ is $0$.  Let's suppose that it's $x$.  The the matrix $L$ is really
\[L = \begin{bmatrix} 0 & 0 \\ y & z \end{bmatrix}, \]
and we still have
\[A = \begin{bmatrix} 0 & 1 \\ 2 & 3 \end{bmatrix} = \begin{bmatrix} 0 & 0 \\ y & z \end{bmatrix}\begin{bmatrix} a & b \\ 0 & c \end{bmatrix}.\]

But now look at the upper right entry of the matrix $A$.  The equation for this is
\[1 = 0 \cdot b  + 0 \cdot c = 0. \]
But this is absurd!  So no matrices $L$ and $U$ can exist.  We'd get the same contradiction (but for $2 \neq 0$) if we had $a =0$.

The problem was the the first column pivot was not in row 1, we had to do a row swap.  If we were finding the $LU$ factorization in order to solve some equations, doing a row swap just corresponds to switching the order of equations, which we can clearly do without changing the solutions.  But in terms of the ordering we choose, an $LU$ factorization does not exist!

\item[5.] Consider the company Widgets-R-Us (WRU).  WRU makes 100 kinds of widgets and sells them in all 50 US states.  Consider the 100 by 50 matrix $A$ with $i,j$-entry recording the number of widgets of type $i$ sold in state $j$ in the month of August 2017.  What kind of 50 by 1 column vectors $x$ could you cook up so that the 100 by 1 column vector $Ax$ records 
\begin{enumerate}
\item the number of each kind of widgets sold in Kansas, 
\item the number of each kind of widget sold in all western states combined, and 
\item the gross revenue obtained from each kind of widget (suppose all widgets have the same price in each state but vary state to state)?  
\end{enumerate}
What kind of 1 by 100 row vectors $y$ could you cook up so that the 1 by 50 row vector $yA$ records 
\begin{enumerate}
\item total widgets sold by state, 
\item total revenue by state (suppose each kind of widget has the same price in all states, but that different widgets can be priced differently)?
\end{enumerate}

\textbf{Solution:}
The matrix $A$ has rows indexed by widgets $w_1, \ldots, w_{100}$ and columns indexed by the states $s_1, \ldots, s_{50}$.
\begin{enumerate}
\item Say that Kansas is the state with index $s_k$.  The we want to extract the $s_k$th column of the matrix $A$.  This is achieved by the elementary vector with $0$ all components $j\neq k$ and a $1$ in the $k$th entry.  So if $k=3$, this would be
\[ \begin{bmatrix} 0 \\ 0 \\ 1 \\ 0 \\ \vdots \\ 0 \end{bmatrix} \]
for example.
\item If the western states have indices $s_{r_1},  \ldots, s_{r_n}$, then the vector should have a $1$ in components $r_1, \ldots, r_n$ and a $0$ elsewhere.
\item Suppose that all widgets in state $s_j$ have price $p_j$.  Then this is achieved by the vector
\[\begin{bmatrix} p_1 \\ p_2 \\ p_3 \\ p_4 \\ \vdots \\ p_{50} \end{bmatrix}. \]
\item[(a)]  To find total widgets by state, we want to sum each column of the matrix.  This is achieved by the matrix
\[\begin{bmatrix} 1 & 1 & \cdots & 1 \end{bmatrix}. \]
\item Suppose that widget $i$ has price $q_i$ in all states.  Then this is achieved by the matrix
\[\begin{bmatrix} q_1 & q_2 & \cdots & q_{100} \end{bmatrix}. \]
\end{enumerate}

\item[6.] (Strang 2.4.31, essentially)  Consider a fixed complex number $A + Bi$, where 
$A, B$ are real.  View complex numbers $z = x +yi$ as column vectors 
\[z = \begin{bmatrix} x \\ y \end{bmatrix} \]
Can you think of a 2 by 2 matrix M such that left multiplication by M is the same as complex multiplication by $A + Bi$?  When is this matrix singular?

\textbf{Solution:}
The product
\[(A + Bi)(x + y i) = (Ax +i^2 By) + (Bx + Ay)i = (Ax - By) + (Bx+Ay)i.\]
Putting this back in vector form, this is represented by
\[\begin{bmatrix} Ax - By \\ Bx + Ay \end{bmatrix} = \begin{bmatrix} A & -B \\ B & A \end{bmatrix} \begin{bmatrix} x \\ y \end{bmatrix}. \]
So the matrix representing ``multiplication by the complex number $A + Bi$" is
\[\begin{bmatrix} A & -B \\ B & A \end{bmatrix}. \]
This is singular if and only if it has fewer than $2$ pivots.  This will happen if (1) $A =0$  and $B = 0$, or if (2) $A \neq 0$ but $A +B^2/A = 0$ (the second pivot is $0$ after clearing the first column).  These two conditions can be combined into one by saying $A^2 + B^2 = 0$.
\end{enumerate}






\end{document}  