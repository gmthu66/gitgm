\documentclass[11pt]{article}
\usepackage[margin=1in]{geometry}                % See geometry.pdf to learn the layout options. There are lots.
\geometry{letterpaper}                   % ... or a4paper or a5paper or ... 
%\geometry{landscape}                % Activate for for rotated page geometry
%\usepackage[parfill]{parskip}    % Activate to begin paragraphs with an empty line rather than an indent


\renewcommand{\aa}{\mathbb{A}}
\newcommand{\cc}{\mathbb{C}}
\newcommand{\rr}{\mathbb{R}}
\newcommand{\pp}{\mathbb{P}}
\newcommand{\hh}{\mathbb{H}}
\newcommand{\qq}{\mathbb{Q}}
\newcommand{\zz}{\mathbb{Z}}
\newcommand{\ff}{\mathbb{F}}
\newcommand{\kk}{\mathbb{K}}
\renewcommand{\gg}{\mathbb{G}}
\newcommand{\nn}{\mathbb{N}}
\renewcommand{\tt}{\mathbb{T}}

\newcommand{\C}{\mathcal{C}}
\newcommand{\U}{\mathcal{U}}
\newcommand{\I}{\mathcal{I}}
\renewcommand{\H}{\mathcal{H}}
\renewcommand{\O}{\mathcal{O}}
\newcommand{\E}{\mathcal{E}}
\newcommand{\F}{\mathcal{F}}
\newcommand{\G}{\mathcal{G}}
\renewcommand{\P}{\mathcal{P}}
\renewcommand{\S}{\mathcal{S}}
\newcommand{\Q}{\mathcal{Q}}
\newcommand{\T}{\mathcal{T}}
\renewcommand{\L}{\mathcal{L}}
\newcommand{\M}{\mathcal{M}}
\newcommand{\X}{\mathcal{X}}

\newcommand{\sH}{\mathscr{H}}
\newcommand{\sD}{\mathscr{D}}
\newcommand{\sE}{\mathscr{E}}
\newcommand{\sL}{\mathscr{L}}
\newcommand{\sQ}{\mathscr{Q}}
\newcommand{\sX}{\mathscr{X}}




\usepackage{graphicx}
\usepackage{amssymb, amsmath}
\usepackage{epstopdf}
\DeclareGraphicsRule{.tif}{png}{.png}{`convert #1 `dirname #1`/`basename #1 .tif`.png}

\title{18.06 Recitation 7}
\author{Isabel Vogt}
\date{\today}                                           % Activate to display a given date or no date

\begin{document}
\maketitle
%\section{}
%\subsection{}
\section{Foundational Problems}

\begin{enumerate}

\item What assumptions on the dimensions of $A$ do we have when finding eigenvalues/eigenvectors of $A$?

\item Find/describe as many eigenvalues and corresponding eigenvectors as you can (without doing any serious calculation) for the following matrices:

\begin{enumerate}

\item $A = \begin{pmatrix} 1 & 0 & 0 \\ 0 & 1 & 0 \\ 0 & 0 & 1 \end{pmatrix}$.

\item $B = \begin{pmatrix} 4 & 3 & -1 \\ 0 & 1 & 4 \\ 0 & 0 &2 \end{pmatrix}$.

\item A projection matrix $P$ some subspace.  If you want to be concrete, think about projection to the column space of 
\[C = \begin{pmatrix} 1 & 1 \\ -1 & 0 \\ 0 & -1 \end{pmatrix}.\]

\item The permutation matrix
\[M = \begin{pmatrix} 0 & 1 & 0 \\ 0 & 0 & 1 \\ 1 & 0 & 0 \end{pmatrix}. \]

\item A rank one matrix $uv^T$.  If you want to be concrete, think about 
\[u = \begin{pmatrix} 1 \\ 2 \\ 0 \end{pmatrix}, \qquad v = \begin{pmatrix} 1 \\ 1 \\ 1 \end{pmatrix}. \]

\end{enumerate}

\item Are there any real eigenvalues of a rotation matrix 
\[R = \begin{pmatrix} \cos \theta & -\sin \theta \\ \sin \theta & \cos\theta \end{pmatrix}, \]
for any possible $\theta$?

\pagebreak

\item Suppose that $A, B, C$ are $m \times m$ matrices with eigenbases that you know.
%\begin{align*}
% v_{A,1}, \cdots, v_{A, m}, & \qquad \lambda_{A, 1}, \cdots, \lambda_{A, m}, \\
% v_{B,1}, \cdots, v_{B, m}, & \qquad \lambda_{B, 1}, \cdots, \lambda_{B, m}, \\
%  v_{C,1}, \cdots, v_{C, m}, & \qquad \lambda_{C, 1}, \cdots, \lambda_{C, m}.
%  \end{align*}
%(So that $A v_{A, i} = \lambda_{A, i} v_{A, i}$, $B v_{B, i} = \lambda_{B, i} v_{B, i}$, and $C v_{C, i} = \lambda_{C, i} v_{C, i}$.)
\begin{enumerate}

\item What do you know about the eigenvectors and eigenvalues of $A^{2017}$?

\item What do you know about the eigenvectors and eigenvalues of $A^{-1}$?

\item What do you know about the eigenvectors and eigenvalues of $A^T$?

\item What do you know about the eigenvectors and eigenvalues of $AB$?

\item What do you know about the eigenvectors and eigenvalues of $A + B$?

\item What do you know about the eigenvectors and eigenvalues of
$\begin{pmatrix} A & C \\ 0 & B \end{pmatrix}$?

\end{enumerate}

\end{enumerate}

\section{Problems}

\begin{enumerate}



%1) what is the pattern when you multiply A repeatedly by some vector?  After ___ multiplications, you get back the same vector, so A^___ = ____
%
%2) What are eigenvalues and eigenvectors of A?
%        --- From the previous part, A^4=I ? is that consistent with your eigenvalues?
%
%3) Write the column vector x = (1,0) in the basis of the eigenvectors and give a formula for A^n x.
%        --- Why is this real even though the eigenvectors/values are complex?
%
%4) What are the eigenvectors and eigenvectors of B = 2A + I?
%        -- can do it both the hard way, of re-solving the characteristic polynomial, and the easy way ? ? 2?+1.
%        -- what does this tell you about B^n x for n goes to +infinity?  To -infinity?
%
%More generally, if A is any real mxm matrix and Ax=?x is a solution with a complex eigenvalue ?, what must be another eigenvalue and eigenvector?  (Answer: complex conjugate of ? and x.)
%
%In general, if Ax=?x and ? is complex, how can you tell whether A^n x blows up as n goes to +infinity?



\item Suppose that $A$ is the matrix
\[A = \begin{pmatrix} 0 & 1 \\ -1 & 0 \end{pmatrix}.\]


\begin{enumerate}

\item What is the pattern when you multiply $A$ repeatedly by some vector?  After \underline{\phantom{aaaaaaa}} multiplications, you get back the same vector, so 
\[A^{\underline{\phantom{aaa}}} = \underline{\phantom{aaaaaaaaaaaa}}.\]

\item What are eigenvalues and eigenvectors of $A$?  Is this consistent with the previous part?

\item Write the vector $x = \begin{pmatrix} 1 \\ 0 \end{pmatrix}$ in the basis of the eigenvectors and give a formula for $A^n x$.

\item What are the eigenvectors and eigenvalues of $B = 2A + I$?

\item What do you know about $B^nx$ as $n \to \infty$ and $n \to -\infty$?

\end{enumerate}

\item Suppose that $A$ is any $m \times m$ matrix with entries in $\rr$ and 
\[Ax = \lambda x \]
for some $\lambda \in \cc, \lambda \not\in \rr$. 
\begin{enumerate}
\item What must be another eigenvector and eigenvalue?
\item How can you tell if $A^n x$ blows up as $n \to \infty$?
\end{enumerate}


\item (Strang, Section 6.1, Problem 19) A $3 \times 3$ matrix $B$ is known to have eigenvalues $0,1,2$.  This is enough information to determine $3$ of the following.  Which are they and what are the answers:
\begin{enumerate}
\item The rank of $B$.
\item The determinant of $B^TB$.
\item The eigenvalues of $B^TB$.
\item The eigenvalues of $(B^2 + I)^{-1}$.
\end{enumerate}

\end{enumerate}
\end{document}  