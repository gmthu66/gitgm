\documentclass[11pt]{article}
\usepackage[margin=1in]{geometry}                % See geometry.pdf to learn the layout options. There are lots.
\geometry{letterpaper}                   % ... or a4paper or a5paper or ... 
%\geometry{landscape}                % Activate for for rotated page geometry
%\usepackage[parfill]{parskip}    % Activate to begin paragraphs with an empty line rather than an indent
\usepackage{graphicx}
\usepackage{amssymb, amsmath}
\usepackage{epstopdf}
\DeclareGraphicsRule{.tif}{png}{.png}{`convert #1 `dirname #1`/`basename #1 .tif`.png}

\title{18.06 Recitation 1}
\author{Isabel Vogt}
\date{September 12, 2017}                                           % Activate to display a given date or no date

\begin{document}
\maketitle
%\section{}
%\subsection{}

\section{Practical Information}
 \begin{enumerate}
 
 \item Stellar is the main website for this class: \texttt{https://stellar.mit.edu/S/course/18/fa17/18.06}
 
 \begin{enumerate}
 
 \item Link to the github where lecture summaries, homework, etc. is posted.
 
 \item Link to Piazza for asking questions.
 
 \item Staff OH under ``Staff" link.  \textbf{My office hours are Tuesdays 8:30-9:30 pm in 2-239.}
 
 \item Link to old OCW videos.  \textit{Warning: Prof. Johnson's lectures can differ substantially from Prof. Strang's!  In particular he goes faster and emphasizes different things; so this is not a substitute for going to lecture!}
 
 \end{enumerate}
 
 \item The syllabus is contained within the slides from the first lecture (on github)
 \begin{enumerate}
 
 \item Midterm 1: September 25
 
 \item Midterm 2: October 30
 
 \item Midterm 3: November 27

\item Final exam announced at the end of this month. 


  
 \item Homework is due every Wednesdays at 11am in my recitation box, \textit{including on exam weeks!}
 
  \end{enumerate}
 
 \item The Math Learning Center is a math-department-run drop-in tutoring service from 3-5pm and 7:30-9:30pm Monday-Thursday (16 hours per week!!) for the classes 18.01, 18.02, 18.03, and 18.06.  This can be a great additional resource for you.
 
 \end{enumerate}

\section{Pictures/Words Problems}

\begin{enumerate}

\item Let $m$ and $n$ be natural numbers (you can think of $m=3$ and $n=2$ if you like).  Let $A$ be an $m \times n$ matrix, $B$ be a $n \times n$ matrix, $v$ be a $1 \times n$ vector, $w$ be a $m \times 1$ vector, and $u$ be a $1 \times m$ vector.  Which of the following make sense (and if so, what is the result)

\begin{enumerate}

\item $A B$

\item $BA$

\item $A v$

\item $A w$

\item $ v A$

\item $v B$

\item $wu$

\item $w v$

\item $u w$

\end{enumerate}

\item If you have two coupled systems of equations
\begin{align*}
Bx + Cy &= c \\
Dx + Ey &= d.
\end{align*}
where $B, C, D$, and $E$ are $3 \times 3$ matrices and $x,y,c$ and $d$ are $3$-component vectors, can you write this as a single system of equations,
\[Az = b, \]
where $z = \begin{bmatrix} x \\ y \end{bmatrix}$ is the $6$-component vector of $x$ on top of $y$.  What are $A$ and $b$?

\item Describe geometrically why the system of linear equations
\begin{align*}
3x + y &= 2\\
6x + 2y &= 4
\end{align*}
has infinitely many solutions.  Describe geometrically why the system of linear equations
\begin{align*}
3x + y &= 2\\
6x + 2y &= 10
\end{align*}
has no solutions.

\item ``Do" Gaussian elimination on the matrix
\[ \begin{bmatrix} 2 & 0 & 0 \\ 4 & 1 & 0 \\ -1 & 6 & 2 \end{bmatrix}.\]

\item What is the $LU$ factorization of the matrix
\[A = \begin{bmatrix} 0 & 1 \\ 2 & 3 \end{bmatrix}? \]

\end{enumerate}

\section{Problems}

\begin{enumerate}

\item Strang 2.2(22): 
\[A = \begin{bmatrix} a & 2 & 3 \\ a& a & 4 \\ a & a & a \end{bmatrix}. \]
For which 3 numbers $a$ will elimination on $A$ fail to give 3 pivots (i.e. for which 3 $a$'s is $A$ singular)?

\item Strang 2.2(27): if you have a lower-triangular system, you can solve it quickly by ``forward substitution".  Solve the $3\times 3$ problem $Lx=b$:

\[\begin{bmatrix} 3 & 0 & 0 \\ 6 & 2 & 0 \\ 9 & -2 & 1 \end{bmatrix} \begin{bmatrix} x \\ y \\ z \end{bmatrix} = \begin{bmatrix} 3 \\ 8 \\ 9 \end{bmatrix}.\]



\item For the lower-triangular matrix $L$ above, find matrices $A$ and $B$ such that
\[A L B = \begin{bmatrix} 1 & -2 & 9 \\ 0 & 2 & 6 \\ 0 & 0 & 3 \end{bmatrix}. \]





\item Suppose that there are a billion Facebook users.  List the billion people 
\[1, 2, ..., \text{billion},\] 
and consider the billion by billion matrix $A$ which has $ij$-entry equal to 1 if people $i$ and $j$ are Facebook friends, and 0 otherwise (say by convention no person is friends with him/herself).  
\begin{enumerate}
\item Can you think of a billion by 1 column vector $x$ such that $Ax$ records how many friends each person has, i.e. so that the $i$th entry of $Ax$ is the number of friends person $i$ has?  
\item Can you think of a billion by 1 column vector $y$ such that $Ay$ records who person 5 is friends with?  
\item Can you think of a billion by 1 column vector $u$ and a 1 by billion row vector $w$ so that the \textit{scalar} $wAu$ records how many friendships there are in the Facebook universe? 
\end{enumerate}

\item Consider the company Widgets-R-Us (WRU).  WRU makes 100 kinds of widgets and sells them in all 50 US states.  Consider the 100 by 50 matrix $A$ with $i,j$-entry recording the number of widgets of type $i$ sold in state $j$ in the month of August 2017.  What kind of 50 by 1 column vectors $x$ could you cook up so that the 100 by 1 column vector $Ax$ records 
\begin{enumerate}
\item the number of each kind of widgets sold in Kansas, 
\item the number of each kind of widget sold in all western states combined, and 
\item the gross revenue obtained from each kind of widget (suppose all widgets have the same price in each state but vary state to state)?  
\end{enumerate}
What kind of 1 by 100 row vectors $y$ could you cook up so that the 1 by 50 row vector $yA$ records 
\begin{enumerate}
\item total widgets sold by state, 
\item total revenue by state (suppose each kind of widget has the same price in all states, but that different widgets can be priced differently)?
\end{enumerate}

\item (Strang 2.4.31, essentially)  Consider a fixed complex number $A + Bi$, where 
$A, B$ are real.  View complex numbers $z = x +yi$ as column vectors 
\[z = \begin{bmatrix} x \\ y \end{bmatrix} \]
Can you think of a 2 by 2 matrix M such that left multiplication by M is the same as complex multiplication by $A + Bi$?  When is this matrix singular?

\end{enumerate}






\end{document}  