\documentclass[11pt]{article}
\usepackage[margin=1in]{geometry}                % See geometry.pdf to learn the layout options. There are lots.
\geometry{letterpaper}                   % ... or a4paper or a5paper or ... 
%\geometry{landscape}                % Activate for for rotated page geometry
%\usepackage[parfill]{parskip}    % Activate to begin paragraphs with an empty line rather than an indent


\renewcommand{\aa}{\mathbb{A}}
\newcommand{\cc}{\mathbb{C}}
\newcommand{\rr}{\mathbb{R}}
\newcommand{\pp}{\mathbb{P}}
\newcommand{\hh}{\mathbb{H}}
\newcommand{\qq}{\mathbb{Q}}
\newcommand{\zz}{\mathbb{Z}}
\newcommand{\ff}{\mathbb{F}}
\newcommand{\kk}{\mathbb{K}}
\renewcommand{\gg}{\mathbb{G}}
\newcommand{\nn}{\mathbb{N}}
\renewcommand{\tt}{\mathbb{T}}

\newcommand{\C}{\mathcal{C}}
\newcommand{\U}{\mathcal{U}}
\newcommand{\I}{\mathcal{I}}
\renewcommand{\H}{\mathcal{H}}
\renewcommand{\O}{\mathcal{O}}
\newcommand{\E}{\mathcal{E}}
\newcommand{\F}{\mathcal{F}}
\newcommand{\G}{\mathcal{G}}
\renewcommand{\P}{\mathcal{P}}
\renewcommand{\S}{\mathcal{S}}
\newcommand{\Q}{\mathcal{Q}}
\newcommand{\T}{\mathcal{T}}
\renewcommand{\L}{\mathcal{L}}
\newcommand{\M}{\mathcal{M}}
\newcommand{\X}{\mathcal{X}}

\newcommand{\sH}{\mathscr{H}}
\newcommand{\sD}{\mathscr{D}}
\newcommand{\sE}{\mathscr{E}}
\newcommand{\sL}{\mathscr{L}}
\newcommand{\sQ}{\mathscr{Q}}
\newcommand{\sX}{\mathscr{X}}




\usepackage{graphicx}
\usepackage{amssymb, amsmath}
\usepackage{epstopdf}
\DeclareGraphicsRule{.tif}{png}{.png}{`convert #1 `dirname #1`/`basename #1 .tif`.png}

\title{18.06 Recitation 4}
\author{Isabel Vogt}
\date{\today}                                           % Activate to display a given date or no date

\begin{document}
\maketitle
%\section{}
%\subsection{}
\section{Pictures/Words Problems}

\begin{enumerate}

\item Describe geometrically in pictures/words these subspaces:

\begin{enumerate}

\item The span of two vectors $v_1, v_2 \in \rr^n$ if $v_1$ and $v_2$ are linearly dependent.

\item The span of three linearly independent vectors $v_1, v_2, v_3 \in \rr^n$.

\item Let $u = \begin{bmatrix} 1 \\ 1 \\ 1 \end{bmatrix}$.  The space of vectors $w \in \rr^3$ so that $u \cdot w = 0$.  Can you describe the geometry of this situation in terms of fundamental subspaces?


\end{enumerate}

\item (Strang, 3.4 Problem 24 + 3.5 Problem 25 + $\epsilon$) True or False (give a good reason if True/example if False)

\begin{enumerate}

\item If the zero vector is in the column space of a matrix $A$, then the columns of $A$ are linearly dependent.

\item If the columns of a matrix are dependent, so are the rows.

\item The column space of a $2 \times 2$ matrix is the same as its row space.

\item The column space of a $2 \times 2$ matrix has the same dimension as its row space.

\item The columns of a matrix are a basis for the column space.

\item $A$ and $A^T$ have the same number of pivots.

\item $A$ and $A^T$ have the same left nullspace.

\item If the row space equals the column space then $A^T=A$.

\item If $A^T = -A$, then the row space of $A$ equals the column space.

\end{enumerate}

\end{enumerate}

\section{Problems}

\begin{enumerate}


\item (Strang 3.4, Problem 7) If $w_1, w_2, w_3$ are independent vectors in $\rr^3$, show that the differences 
\begin{align*}
v_1 &= w_2 - w_3 \\
v_2 &= w_1 - w_3 \\
v_3 &= w_1 - w_2.
\end{align*}
are \emph{dependent}.  Find the matrix $A$ so that
\[ \begin{bmatrix} \\ w_1 & w_2 & w_3 \\ &\end{bmatrix}A = \begin{bmatrix} \\ v_1 & v_2 & v_3 \\ &\end{bmatrix}.\]
Which matrices above are singular?

\item \textbf{(Strang 3.4, Problem 11)} $A$ is an $m \times n$ matrix of rank $r$.  Suppose there are right sides $b$ for which $Ax = b$ has no solution.  (What does this mean in terms of fundamental subspaces?)


\begin{enumerate}

\item What are all of the inequalities ($<$ or $\leq$) that must be true between $m$ and $n$ and $r$?

\item How do you know that $A^Ty=0$ has solutions other than $y=0$?


\end{enumerate}

\item \textbf{(Strang 3.4, Problem 22)} Construct $A = uv^T + wz^T$ whose column space has basis $\begin{bmatrix} 1 \\ 2 \\ 4 \end{bmatrix}, \begin{bmatrix} 2 \\ 2 \\ 1 \end{bmatrix}$ and whose row space has basis $(1,0), (1,1)$.  Write $A$ as a $3 \times 2$ matrix times a $2 \times 2$ matrix.

\item (Strang 3.5, Problem 5) 
\begin{enumerate}

\item If $Ax = b$ has a solution and $A^Ty = 0$, is
\[ y^Tx = 0, \qquad \text{or} \qquad y^Tb = 0? \]

\item If $A^Ty = (1,1,1)$ has a solution and $Ax = 0$, then \underline{\phantom{aaaaaaaaaaaaaa}}.

\end{enumerate}

\item  \textbf{(Strang 3.5, Problem 23)} If a subspace $S$ is contained in a subspace $V$, prove that $S^\perp$ contains $V^\perp$.

\item \textbf{(Strang 3.5, Problem 27)} The lines $3x + y = b_1$ and $6x + 2y = b_2$ are \underline{\phantom{aaaaaaaaaaaaaa}}.  They are the same line if \underline{\phantom{aaaaaaaaaaaaaa}}. In that case $(b_1, b_2)$ is perpendicular to the vector \underline{\phantom{aaaaaaaaaaaaaa}}.  The nullspace of the matrix is the line $3x + y =$ \underline{\phantom{aaaaaaaa}}.  One particular vector in the nullspace is \underline{\phantom{aaaaaaaaaaaaaa}}.

\item (Strang 3.5, Problem 33)** Suppose I give you $8$ vectors $r_1, r_2, n_1, n_2, c_1, c_2, l_1, l_2$ in $\rr^4$.

\begin{enumerate}

\item What are the conditions for those pairs to be bases for the our fundamental subspaces of a $4 \times 4$ matrix?

\item What is one possible matrix A?

\end{enumerate}




\end{enumerate}
\end{document}  