\documentclass[11pt]{article}
\usepackage[margin=1in]{geometry}                % See geometry.pdf to learn the layout options. There are lots.
\geometry{letterpaper}                   % ... or a4paper or a5paper or ... 
%\geometry{landscape}                % Activate for for rotated page geometry
%\usepackage[parfill]{parskip}    % Activate to begin paragraphs with an empty line rather than an indent


\renewcommand{\aa}{\mathbb{A}}
\newcommand{\cc}{\mathbb{C}}
\newcommand{\rr}{\mathbb{R}}
\newcommand{\pp}{\mathbb{P}}
\newcommand{\hh}{\mathbb{H}}
\newcommand{\qq}{\mathbb{Q}}
\newcommand{\zz}{\mathbb{Z}}
\newcommand{\ff}{\mathbb{F}}
\newcommand{\kk}{\mathbb{K}}
\renewcommand{\gg}{\mathbb{G}}
\newcommand{\nn}{\mathbb{N}}
\renewcommand{\tt}{\mathbb{T}}

\newcommand{\C}{\mathcal{C}}
\newcommand{\U}{\mathcal{U}}
\newcommand{\I}{\mathcal{I}}
\renewcommand{\H}{\mathcal{H}}
\renewcommand{\O}{\mathcal{O}}
\newcommand{\E}{\mathcal{E}}
\newcommand{\F}{\mathcal{F}}
\newcommand{\G}{\mathcal{G}}
\renewcommand{\P}{\mathcal{P}}
\renewcommand{\S}{\mathcal{S}}
\newcommand{\Q}{\mathcal{Q}}
\newcommand{\T}{\mathcal{T}}
\renewcommand{\L}{\mathcal{L}}
\newcommand{\M}{\mathcal{M}}
\newcommand{\X}{\mathcal{X}}

\newcommand{\sH}{\mathscr{H}}
\newcommand{\sD}{\mathscr{D}}
\newcommand{\sE}{\mathscr{E}}
\newcommand{\sL}{\mathscr{L}}
\newcommand{\sQ}{\mathscr{Q}}
\newcommand{\sX}{\mathscr{X}}




\usepackage{graphicx}
\usepackage{amssymb, amsmath}
\usepackage{epstopdf}
\DeclareGraphicsRule{.tif}{png}{.png}{`convert #1 `dirname #1`/`basename #1 .tif`.png}

%\title{18.06 Recitation 6}
%\author{Isabel Vogt}
%\date{\today}                                           % Activate to display a given date or no date

\begin{document}
%\maketitle
%\section{}
%\subsection{}


\section{Problems}

\begin{enumerate}

\item (Spring 2009, Exam 1, Problem 3)  $A$ is a matrix with nullspace $N(A)$ spanned by the following $3$ vectors
\[\begin{pmatrix} 1 \\ 2 \\ -1 \\ 3 \end{pmatrix}, \begin{pmatrix} 0 \\ 1 \\ 1 \\ 4 \end{pmatrix},\begin{pmatrix} -1 \\ -1 \\ 3 \\ 1 \end{pmatrix}\]
\begin{enumerate}

\item Give a matrix $B$ so that its column space $C(B)$ is the same as $N(A)$.

\item Give a different possible answer to (a): another $B$ with $C(B) = N(A)$.

\item For some vector $\mathbf{b}$, you are told that a particular solution to $A \mathbf{x} = \mathbf{b}$ is
\[\mathbf{x}_p = \begin{pmatrix} 1 \\ 2 \\ 3 \\ 4 \end{pmatrix}. \]
Now your classmate Zarkon tells you that a second solution is
\[\mathbf{x}_Z = \begin{pmatrix} 1 \\ 1 \\ 3 \\ 0 \end{pmatrix},\]
while your other classmate Hastur tells you ``No, Zarkon's solution can't be right, but here's a second solution that is correct."
\[\mathbf{x}_H = \begin{pmatrix} 1 \\ 1 \\ 3 \\ 1 \end{pmatrix}.\]
Is Zarkon's solution correct, or Hastur's solution, or are both correct?

\end{enumerate}

\item (essentially Fall 2005, Exam 2, \#9 from 2009 practice problems)  We take measurements $1,4,b_3$ at times $t = 1,2,3$.  We want to find the nearest line $C + Dt$ using least squares.
\begin{enumerate}

\item Which value of $b_3$ will put all three points on the same line.  Give $C$ and $D$ for this line.

\item Will least squares choose this line if $b_3 = 9$?  

\item What is the linear system $Ax = b$ that would be solved exactly for $x = (C, D)$ if all three points lie on a line.  Write down a formula for the projection matrix onto the column space of $A$.  What are the dimensions of the matrix?  What is its rank?  What is its column space?

\item More generally, what is the linear system $Gx = f$ that is solved exactly by the least squares solution $\hat{x} = (\hat{C}, \hat{D})$.

\end{enumerate}

\item (Fall 2012, Exam 2, Problem 1) $P$ is any $n \times n$ projection matrix.  Compute the ranks of $A, B$, and $C$ below.  Your method must be visibly correct for every such $P$, not just one example.
\begin{enumerate}

\item $A = (I-P)P$.

\item $B = (I-P) - P$.  (Hint: squaring $B$ might be helpful.)

\item $C = (I-P)^{2017} + P^{2017}$.

\end{enumerate}

\item (Fall 2012, Exam 2, Problem 3) The $3 \times 3$ matrix $\begin{pmatrix} a & b & c \\ d & e & f \\ g & h & i \end{pmatrix}$ has $QR$ decomposition
\[\begin{pmatrix} a & b & c \\ d & e & f \\ g & h & i \end{pmatrix} = Q \begin{pmatrix} r_{11} & r_{12} & r_{13} \\ 0 & r_{22} & r_{23} \\ 0 & 0 & r_{33} \end{pmatrix}. \]
\begin{enumerate}

\item What is $r_{11}$ in terms of $a,b,c,d,e,f,g,h,i$? (but not any of the elements of $Q$!)


\item Solve for $x$ in the equation
\[ Q^T x = \begin{pmatrix} 1 \\ 0 \\ 0 \end{pmatrix}, \]
expressing your solution possibly in terms of $r_{11}, r_{22}, r_{33}$ and $a,b,c,d,e,f,g,h,i$ (but again not any of the elements of $Q$.)

\end{enumerate}







\end{enumerate}
\end{document}  